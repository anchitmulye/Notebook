\section{Data Structures}

\subsection{Write a program to implement a stack (using arrays or linked lists).}
\begin{tcolorbox}[title=]
\begin{verbatim}
#define MAX_SIZE 100

class Stack {
private:
    int arr[MAX_SIZE];
    int top;

public:
    Stack() {
        top = -1;
    }

    void push(int value) {
        if (top >= MAX_SIZE - 1) {
            cout << "Stack Overflow\n";
            return;
        }
        arr[++top] = value;
    }

    int pop() {
        if (top < 0) {
            cout << "Stack Underflow\n";
            return -1; // or throw an exception
        }
        return arr[top--];
    }

    int peek() {
        if (top < 0) {
            cout << "Stack is empty\n";
            return -1; // or throw an exception
        }
        return arr[top];
    }

    bool isEmpty() {
        return (top == -1);
    }
};

int main() {
    Stack s;
    s.push(10);
    s.push(20);
    s.push(30);
    cout << s.pop() << " popped from stack\n";
    cout << "Top element is " << s.peek() << endl;
}    
\end{verbatim}
\end{tcolorbox}

\subsection{Write a program to implement a queue (using arrays or linked lists).}
\begin{tcolorbox}[title=]
\begin{verbatim}
#define MAX_SIZE 100

class Queue {
private:
    int arr[MAX_SIZE];
    int front, rear;

public:
    Queue() {
        front = -1;
        rear = -1;
    }

    void enqueue(int value) {
        if (rear >= MAX_SIZE - 1) {
            cout << "Queue Overflow\n";
            return;
        }
        arr[++rear] = value;
        if (front == -1) {
            front = 0;
        }
    }

    int dequeue() {
        if (front == -1 || front > rear) {
            cout << "Queue Underflow\n";
            return -1; // or throw an exception
        }
        return arr[front++];
    }

    int peek() {
        if (front == -1 || front > rear) {
            cout << "Queue is empty\n";
            return -1; // or throw an exception
        }
        return arr[front];
    }

    bool isEmpty() {
        return (front == -1 || front > rear);
    }
};  
\end{verbatim}
\end{tcolorbox}

\subsection{Write a program to implement a binary search tree and perform insertions and deletions.}
\begin{tcolorbox}[title=]
\begin{verbatim}
    struct TreeNode {
        int data;
        TreeNode* left;
        TreeNode* right;
        TreeNode(int val) : data(val), left(nullptr), right(nullptr) {}
    };
    
    class BinarySearchTree {
    private:
        TreeNode* root;
        TreeNode* insertRecursive(TreeNode* node, int key) {
            if (node == nullptr) {
                return new TreeNode(key);
            }
            if (key < node->data) {
                node->left = insertRecursive(node->left, key);
            } else if (key > node->data) {
                node->right = insertRecursive(node->right, key);
            }
            return node;
        }
    
        TreeNode* minValueNode(TreeNode* node) {
            TreeNode* current = node;
            while (current && current->left != nullptr) {
                current = current->left;
            }
            return current;
        }
    
        TreeNode* deleteRecursive(TreeNode* node, int key) {
            if (node == nullptr) {
                return node;
            }
            if (key < node->data) {
                node->left = deleteRecursive(node->left, key);
            } else if (key > node->data) {
                node->right = deleteRecursive(node->right, key);
            } else {
                if (node->left == nullptr) {
                    TreeNode* temp = node->right;
                    delete node;
                    return temp;
                } else if (node->right == nullptr) {
                    TreeNode* temp = node->left;
                    delete node;
                    return temp;
                }
                TreeNode* temp = minValueNode(node->right);
                node->data = temp->data;
                node->right = deleteRecursive(node->right, temp->data);
            }
            return node;
        }
\end{verbatim}
\end{tcolorbox}
\begin{tcolorbox}
\begin{verbatim}
    public:
        BinarySearchTree() : root(nullptr) {}
    
        void insert(int key) {
            root = insertRecursive(root, key);
        }
    
        void remove(int key) {
            root = deleteRecursive(root, key);
        }
    
        void inorderTraversal(TreeNode* node) {
            if (node == nullptr) {
                return;
            }
            inorderTraversal(node->left);
            cout << node->data << " ";
            inorderTraversal(node->right);
        }
    
        void inorder() {
            inorderTraversal(root);
            cout << endl;
        }
    };
    
    int main() {
    BinarySearchTree bst;
    bst.insert(50);
    bst.insert(30);
    bst.insert(20);
    bst.insert(40);
    bst.insert(70);
    bst.insert(60);
    bst.insert(80);

    cout << "Inorder traversal of BST: ";
    bst.inorder();

    bst.remove(20);
    cout << "Inorder traversal after deleting 20: ";
    bst.inorder();

    bst.remove(30);
    cout << "Inorder traversal after deleting 30: ";
    bst.inorder();

    bst.remove(50);
    cout << "Inorder traversal after deleting 50: ";
    bst.inorder();

    return 0;
}
\end{verbatim}
\end{tcolorbox}

\subsection{Write a program to implement a linked list and perform insertions and deletions.}
\begin{tcolorbox}[title=]
\begin{verbatim}
    #include <iostream>
    using namespace std;
    
    struct ListNode {
        int data;
        ListNode* next;
        ListNode(int val) : data(val), next(nullptr) {}
    };
    
    class LinkedList {
    private:
        ListNode* head;
    
    public:
        LinkedList() : head(nullptr) {}
    
        void insert(int val) {
            ListNode* newNode = new ListNode(val);
            newNode->next = head;
            head = newNode;
        }
    
        void remove(int val) {
            ListNode* current = head;
            ListNode* prev = nullptr;
    
            while (current != nullptr && current->data != val) {
                prev = current;
                current = current->next;
            }
    
            if (current == nullptr) {
                cout << "Element " << val << " not found in the list." << endl;
                return;
            }
    
            if (prev == nullptr) {
                head = current->next;
            } else {
                prev->next = current->next;
            }
    
            delete current;
        }
    
        void display() {
            ListNode* current = head;
            while (current != nullptr) {
                cout << current->data << " ";
                current = current->next;
            }
            cout << endl;
        }
    };
\end{verbatim}
\end{tcolorbox}

\subsection{Write a program to implement a basic queue using two stacks.}
\begin{tcolorbox}[title=]
\begin{verbatim}
    #include <iostream>
    #include <stack>
    using namespace std;
    
    class Queue {
    private:
        stack<int> s1;  // main stack for enqueue
        stack<int> s2;  // temporary stack for dequeue
    
    public:
        void enqueue(int val) {
            s1.push(val);
        }
    
        int dequeue() {
            if (s1.empty() && s2.empty()) {
                cout << "Queue is empty." << endl;
                return -1;  // indicating queue is empty
            }
    
            if (s2.empty()) {
                // Transfer all elements from s1 to s2
                while (!s1.empty()) {
                    s2.push(s1.top());
                    s1.pop();
                }
            }
    
            int front = s2.top();
            s2.pop();
            return front;
        }
    
        bool isEmpty() {
            return s1.empty() && s2.empty();
        }
    };
    
    int main() {
        Queue q;
        q.enqueue(1);
        q.enqueue(2);
        q.enqueue(3);
        cout << "Dequeued element: " << q.dequeue() << endl;
        cout << "Dequeued element: " << q.dequeue() << endl;
        q.enqueue(4);
        q.enqueue(5);
        while (!q.isEmpty()) {
            cout << "Dequeued element: " << q.dequeue() << endl;
        }
        cout << "Dequeued element: " << q.dequeue() << endl; // Dequeue from empty
    }
\end{verbatim}
\end{tcolorbox}
