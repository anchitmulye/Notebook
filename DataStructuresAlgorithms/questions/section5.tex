\section{Algorithms}

\subsection{Write a program to multiply two matrices.}
\begin{tcolorbox}[title=]
\begin{verbatim}
const int N = 3;

void multiplyMatrices(int mat1[][N], int mat2[][N], int result[][N]) {
    for (int i = 0; i < N; ++i) {
        for (int j = 0; j < N; ++j) {
            result[i][j] = 0;
            for (int k = 0; k < N; ++k) {
                result[i][j] += mat1[i][k] * mat2[k][j];
            }
        }
    }
}    
\end{verbatim}
\end{tcolorbox}

\subsection{Write a program to implement a linear search algorithm.}
\begin{tcolorbox}[title=]
\begin{verbatim}
int linearSearch(int arr[], int n, int key) {
    for (int i = 0; i < n; ++i) {
        if (arr[i] == key) {
            return i; // Return index of key if found
        }
    }
    return -1; // Return -1 if key is not found
}
\end{verbatim}
\end{tcolorbox}

\subsection{Write a program to implement a binary search algorithm.}
\begin{tcolorbox}[title=]
\begin{verbatim}
int binarySearch(int arr[], int l, int r, int key) {
    while (l <= r) {
        int mid = l + (r - l) / 2;
        if (arr[mid] == key) {
            return mid; // Return index of key if found
        }
        if (arr[mid] < key) {
            l = mid + 1; // Search in the right half
        } else {
            r = mid - 1; // Search in the left half
        }
    }
    return -1; // Return -1 if key is not found
}
\end{verbatim}
\end{tcolorbox}

\subsection{Write a program to implement a bubble sort algorithm.}
\begin{tcolorbox}[title=]
\begin{verbatim}
void bubbleSort(int arr[], int n) {
    for (int i = 0; i < n - 1; ++i) {
        for (int j = 0; j < n - i - 1; ++j) {
            if (arr[j] > arr[j + 1]) {
                // Swap arr[j] and arr[j + 1]
                int temp = arr[j];
                arr[j] = arr[j + 1];
                arr[j + 1] = temp;
            }
        }
    }
}
\end{verbatim}
\end{tcolorbox}

\subsection{Write a program to implement a selection sort algorithm.}
\begin{tcolorbox}[title=]
\begin{verbatim}
void selectionSort(int arr[], int n) {
    for (int i = 0; i < n - 1; ++i) {
        int min_idx = i;
        for (int j = i + 1; j < n; ++j) {
            if (arr[j] < arr[min_idx]) {
                min_idx = j;
            }
        }
        // Swap arr[i] and arr[min_idx]
        int temp = arr[i];
        arr[i] = arr[min_idx];
        arr[min_idx] = temp;
    }
}
\end{verbatim}
\end{tcolorbox}

\subsection{Write a program to implement an insertion sort algorithm.}
\begin{tcolorbox}[title=]
\begin{verbatim}
void insertionSort(int arr[], int n) {
    for (int i = 1; i < n; ++i) {
        int key = arr[i];
        int j = i - 1;
        while (j >= 0 && arr[j] > key) {
            arr[j + 1] = arr[j];
            j--;
        }
        arr[j + 1] = key;
    }
}
\end{verbatim}
\end{tcolorbox}

\subsection{Write a program to implement an quick sort algorithm.}
\begin{tcolorbox}[title=]
\begin{verbatim}
int partition(vector<int>& arr, int low, int high) {
    int pivot = arr[high];  // Choosing the last element as the pivot
    int i = low - 1;  // Index of the smaller element

    for (int j = low; j < high; ++j) {
        if (arr[j] < pivot) {
            ++i;
            swap(arr[i], arr[j]);
        }
    }

    swap(arr[i + 1], arr[high]);
    return (i + 1);
}

// Function to implement Quick Sort
void quickSort(vector<int>& arr, int low, int high) {
    if (low < high) {
        int pi = partition(arr, low, high);

        quickSort(arr, low, pi - 1);  // Sort elements before partition
        quickSort(arr, pi + 1, high); // Sort elements after partition
    }
}
\end{verbatim}
\end{tcolorbox}

\subsection{Write a program to implement an merge sort algorithm.}
\begin{tcolorbox}[title=]
\begin{verbatim}
void merge(vector<int>& arr, int left, int mid, int right) {
    int n1 = mid - left + 1;
    int n2 = right - mid;

    // Create temporary arrays
    vector<int> L(n1), R(n2);

    // Copy data to temporary arrays L[] and R[]
    for (int i = 0; i < n1; ++i)
        L[i] = arr[left + i];
    for (int j = 0; j < n2; ++j)
        R[j] = arr[mid + 1 + j];

    // Merge the temporary arrays back into arr[left..right]
    int i = 0;  // Initial index for left subarray
    int j = 0;  // Initial index for right subarray
    int k = left;  // Initial index for merged subarray

    while (i < n1 && j < n2) {
        if (L[i] <= R[j]) {
            arr[k] = L[i];
            ++i;
        } else {
            arr[k] = R[j];
            ++j;
        }
        ++k;
    }

    // Copy the remaining elements of L[], if any
    while (i < n1) {
        arr[k] = L[i];
        ++i;
        ++k;
    }

    // Copy the remaining elements of R[], if any
    while (j < n2) {
        arr[k] = R[j];
        ++j;
        ++k;
    }
}

// Function to implement Merge Sort
void mergeSort(vector<int>& arr, int left, int right) {
    if (left < right) {
        int mid = left + (right - left) / 2; 
        mergeSort(arr, left, mid);  // Sort left half
        mergeSort(arr, mid + 1, right); // Sort right half
        merge(arr, left, mid, right); // Merge the sorted halves
    }
}
\end{verbatim}
\end{tcolorbox}

\subsection{Write a program to find the largest sum contiguous subarray (Kadane’s Algorithm).}
\begin{tcolorbox}[title=]
\begin{verbatim}
int kadane(int arr[], int n) {
    int max_so_far = arr[0];
    int max_ending_here = arr[0];

    for (int i = 1; i < n; ++i) {
        max_ending_here = max(arr[i], max_ending_here + arr[i]);
        max_so_far = max(max_so_far, max_ending_here);
    }

    return max_so_far;
}
\end{verbatim}
\end{tcolorbox}