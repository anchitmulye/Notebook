\section{Numbers}

\subsection{Write a program to find the largest number among three numbers.}
\begin{tcolorbox}[title=]
\begin{verbatim}
int findLargest(int a, int b, int c) {
    // Compare the three numbers and return the largest one
    if (a >= b && a >= c) {
        return a;
    } else if (b >= a && b >= c) {
        return b;
    } else {
        return c;
    }
}
\end{verbatim}
\end{tcolorbox}

\subsection{Write a program to check if a number is even or odd.}
\begin{tcolorbox}[title=]
\begin{verbatim}
bool isEven(int num) {
    // A number is even if it is divisible by 2
    return num % 2 == 0;
}
\end{verbatim}
\end{tcolorbox}

\subsection{Write a program to check if a year is a leap year.}
\begin{tcolorbox}[title=]
\begin{verbatim}
bool isLeapYear(int year) {
    // A year is a leap year if it is divisible by 4
    // but not divisible by 100 unless it is also divisible by 400
    if (year % 4 == 0) {
        if (year % 100 == 0) {
            if (year % 400 == 0) {
                return true;
            } else {
                return false;
            }
        } else {
            return true;
        }
    } else {
        return false;
    }
}
\end{verbatim}
\end{tcolorbox}

\subsection{Write a program to swap two numbers.}
\begin{tcolorbox}[title=]
\begin{verbatim}
void swapNumbers(int &a, int &b) {
    // Swap the values of a and b using a temporary variable
    int temp = a;
    a = b;
    b = temp;
}
\end{verbatim}
\end{tcolorbox}

\subsection{Write a program to find the factorial of a number.}
\begin{tcolorbox}[title=]
\begin{verbatim}
int factorial(int n) {
    // Initialize result to 1
    int result = 1;
    // Multiply result by each number from 1 to n
    for (int i = 1; i <= n; i++) {
        result *= i;
    }
    return result;
}
\end{verbatim}
\end{tcolorbox}

\subsection{Write a program to check if a number is prime.}
\begin{tcolorbox}[title=]
\begin{verbatim}
bool isPrime(int num) {
    // A number less than or equal to 1 is not prime
    if (num <= 1) return false;
    // Check divisibility from 2 to the square root of num
    for (int i = 2; i * i <= num; i++) {
        if (num % i == 0) return false;
    }
    return true;
}
\end{verbatim}
\end{tcolorbox}


\subsection{Write a program to print the Fibonacci series up to a certain number of terms.}
\begin{tcolorbox}[title=]
\begin{verbatim}
void printFibonacci(int n) {
    // First two terms
    int a = 0, b = 1;
    cout << a << " " << b << " ";
    for (int i = 2; i < n; i++) {
        int next = a + b;
        cout << next << " ";
        a = b;
        b = next;
    }
}
\end{verbatim}
\end{tcolorbox}


\subsection{Write a program to reverse a number.}
\begin{tcolorbox}[title=]
\begin{verbatim}  
int reverseNumber(int num) {
    int reversed = 0;
    while (num > 0) {
        reversed = reversed * 10 + num % 10;
        num /= 10;
    }
    return reversed;
}
\end{verbatim}
\end{tcolorbox}


\subsection{Write a program to check if a number is Armstrong number.}
\begin{tcolorbox}[title=]
\begin{verbatim}
bool isArmstrong(int num) {
    int originalNum = num;
    int sum = 0;
    int n = to_string(num).length(); // Number of digits in num
    while (num > 0) {
        int digit = num % 10;
        sum += pow(digit, n);
        num /= 10;
    }
    return sum == originalNum;
}
\end{verbatim}
\end{tcolorbox}


\subsection{Write a program to find the GCD (Greatest Common Divisor) of two numbers.}
\begin{tcolorbox}[title=]
\begin{verbatim}
int gcd(int a, int b) {
    // Use the Euclidean algorithm
    while (b != 0) {
        int temp = b;
        b = a % b;
        a = temp;
    }
    return a;
}
\end{verbatim}
\end{tcolorbox}


\subsection{Write a program to find the LCM (Least Common Multiple) of two numbers.}
\begin{tcolorbox}[title=]
\begin{verbatim}
int lcm(int a, int b) {
    return (a * b) / gcd(a, b); // Using LCM(a, b) * GCD(a, b) = a * b
}
\end{verbatim}
\end{tcolorbox}


\subsection{Write a program to find the factorial of a number using recursion.}
\begin{tcolorbox}[title=]
\begin{verbatim}
int factorialRecursive(int n) {
    if (n == 0) return 1; // Base case
    return n * factorialRecursive(n - 1); // Recursive case
}
\end{verbatim}
\end{tcolorbox}


\subsection{Write a program to find the nth Fibonacci number using recursion.}
\begin{tcolorbox}[title=]
\begin{verbatim}
int fibonacciRecursive(int n) {
    if (n <= 1) return n; // Base case
    return fibonacciRecursive(n - 1) + fibonacciRecursive(n - 2);
}
\end{verbatim}
\end{tcolorbox}


\subsection{Write a program to check if a number is a perfect number.}
\begin{tcolorbox}[title=]
\begin{verbatim} 
bool isPerfectNumber(int num) {
    int sum = 1;
    for (int i = 2; i * i <= num; i++) {
        if (num % i == 0) {
            if (i * i != num) {
                sum += i + num / i;
            } else {
                sum += i;
            }
        }
    }
    return sum == num && num != 1;
}
\end{verbatim}
\end{tcolorbox}


\subsection{Write a program to convert decimal to binary.}
\begin{tcolorbox}[title=]
\begin{verbatim}
string decimalToBinary(int num) {
    string binary = "";
    while (num > 0) {
        binary = to_string(num % 2) + binary;
        num /= 2;
    }
    return binary;
}
\end{verbatim}
\end{tcolorbox}


\subsection{Write a program to convert binary to decimal.}
\begin{tcolorbox}[title=]
\begin{verbatim}
int binaryToDecimal(string binary) {
    int decimal = 0;
    int base = 1;
    for (int i = binary.length() - 1; i >= 0; i--) {
        if (binary[i] == '1') {
            decimal += base;
        }
        base *= 2;
    }
    return decimal;
}
\end{verbatim}
\end{tcolorbox}


\subsection{Write a program to check if a number is a palindrome using recursion.}
\begin{tcolorbox}[title=]
\begin{verbatim}
bool isPalindromeRecursive(string str, int start, int end) {
    if (start >= end) return true; // Base case
    if (str[start] != str[end]) return false;
    return isPalindromeRecursive(str, start + 1, end - 1); // Recursive case
}
\end{verbatim}
\end{tcolorbox}


\subsection{Write a program to check if a number is a power of two.}
\begin{tcolorbox}[title=]
\begin{verbatim}
bool isPowerOfTwo(int num) {
    // A number is a power of two if it has exactly one bit 
    // set in its binary representation
    return (num > 0) && ((num & (num - 1)) == 0);
}
\end{verbatim}
\end{tcolorbox}


\subsection{Write a program to find the factorial of a number using an iterative approach.}
\begin{tcolorbox}[title=]
\begin{verbatim}
int factorialIterative(int n) {
    int result = 1;
    for (int i = 1; i <= n; i++) {
        result *= i;
    }
    return result;
}
\end{verbatim}
\end{tcolorbox}


\subsection{Write a program to check if a number is a perfect square.}
\begin{tcolorbox}[title=]
\begin{verbatim}
bool isPerfectSquare(int num) {
    int sqrtNum = sqrt(num);
    return (sqrtNum * sqrtNum == num);
}
\end{verbatim}
\end{tcolorbox}


\subsection{Write a program to convert a decimal number to its Roman numeral equivalent.}
\begin{tcolorbox}[title=]
\begin{verbatim}
string decimalToRoman(int num) {
    // Arrays of roman numerals and their corresponding values
    string romans[] = {"M", "CM", "D", "CD", "C", "XC", "L", "XL",
                         "X", "IX", "V", "IV", "I"};
    int values[] = {1000, 900, 500, 400, 100, 90, 50, 40, 10, 9, 5, 4, 1};
    string roman = "";
    
    for (int i = 0; i < 13; i++) {
        while (num >= values[i]) {
            num -= values[i];
            roman += romans[i];
        }
    }
    return roman;
}
\end{verbatim}
\end{tcolorbox}


\subsection{Write a program to find the next prime number greater than a given number.}
\begin{tcolorbox}[title=]
\begin{verbatim}
int nextPrime(int num) {
    num++;
    while (!isPrime(num)) {
        num++;
    }
    return num;
}
\end{verbatim}
\end{tcolorbox}



































